% ***********************************************************
% ******************* PHYSICS HEADER ************************
% ***********************************************************
% Version 2
\documentclass[12pt]{article}
\usepackage{amsmath} % AMS Math Package
\usepackage{amsthm} % Theorem Formatting
\usepackage{amssymb}    % Math symbols such as \mathbb
\usepackage{graphicx} % Allows for eps images
\usepackage[dvips,letterpaper,margin=1in,bottom=0.7in]{geometry}
\usepackage{tensor}
 % Sets margins and page size
\renewcommand{\labelenumi}{(\alph{enumi})} % Use letters for enumerate
% \DeclareMathOperator{\Sample}{Sample}
\let\vaccent=\v % rename builtin command \v{} to \vaccent{}
\usepackage{enumerate}
\renewcommand{\v}[1]{\ensuremath{\mathbf{#1}}} % for vectors
\newcommand{\gv}[1]{\ensuremath{\mbox{\boldmath$ #1 $}}} 
% for vectors of Greek letters
\newcommand{\uv}[1]{\ensuremath{\mathbf{\hat{#1}}}} % for unit vector
\newcommand{\abs}[1]{\left| #1 \right|} % for absolute value
\newcommand{\avg}[1]{\left< #1 \right>} % for average
\let\underdot=\d % rename builtin command \d{} to \underdot{}
\renewcommand{\d}[2]{\frac{d #1}{d #2}} % for derivatives
\newcommand{\dd}[2]{\frac{d^2 #1}{d #2^2}} % for double derivatives
\newcommand{\pd}[2]{\frac{\partial #1}{\partial #2}} 
% for partial derivatives
\newcommand{\pdd}[2]{\frac{\partial^2 #1}{\partial #2^2}} 
% for double partial derivatives
\newcommand{\pdc}[3]{\left( \frac{\partial #1}{\partial #2}
 \right)_{#3}} % for thermodynamic partial derivatives
\newcommand{\ket}[1]{\left| #1 \right>} % for Dirac bras
\newcommand{\bra}[1]{\left< #1 \right|} % for Dirac kets
\newcommand{\braket}[2]{\left< #1 \vphantom{#2} \right|
 \left. #2 \vphantom{#1} \right>} % for Dirac brackets
\newcommand{\matrixel}[3]{\left< #1 \vphantom{#2#3} \right|
 #2 \left| #3 \vphantom{#1#2} \right>} % for Dirac matrix elements
\newcommand{\grad}[1]{\gv{\nabla} #1} % for gradient
\let\divsymb=\div % rename builtin command \div to \divsymb
\renewcommand{\div}[1]{\gv{\nabla} \cdot \v{#1}} % for divergence
\newcommand{\curl}[1]{\gv{\nabla} \times \v{#1}} % for curl
\let\baraccent=\= % rename builtin command \= to \baraccent
\renewcommand{\=}[1]{\stackrel{#1}{=}} % for putting numbers above =
\providecommand{\wave}[1]{\v{\tilde{#1}}}
\providecommand{\fr}{\frac}
\providecommand{\RR}{\mathbb{R}}
\providecommand{\NN}{\mathbb{N}}
\providecommand{\seq}{\subseteq}
\providecommand{\e}{\epsilon}

\newtheorem{prop}{Proposition}
\newtheorem{thm}{Theorem}[section]
\newtheorem{axiom}{Axiom}[section]
\newtheorem{p}{Problem}[section]
\usepackage{cancel}
\newtheorem*{lem}{Lemma}
\theoremstyle{definition}
\newtheorem*{dfn}{Definition}
 \newenvironment{s}{%\small%
        \begin{trivlist} \item \textbf{Solution}. }{%
            \hspace*{\fill} $\blacksquare$\end{trivlist}}%
% ***********************************************************
% ********************** END HEADER *************************
% ***********************************************************

\begin{document}

{\noindent\Huge\bf  \\[0.5\baselineskip] {\fontfamily{cmr}\selectfont  Problem Set I}         }\\[2\baselineskip] % Title
{ {\bf \fontfamily{cmr}\selectfont Computing Models}\\ {\textit{\fontfamily{cmr}\selectfont     April 23, 2023}}}~~~~~~~~~~~~~~~~~~~~~~~~~~~~~~~~~~~~~~~~~~~~~~~~~~~~~~~~~~~~~~~~~~~~~~~~~~~~~    {\large \textsc{Alon Filler}\footnote{With $\Sigma$orer}} % Author name
\\[1.4\baselineskip] 



\section{Automatas}
\emph{Given the automata $D = (Q^{D}, \{a,b\}, \delta^{D}, q_0^{D}, F^{D})$} \newline
\begin{p}
\emph{\newline Imagine a new automata $E = (Q^{E}, \{a,b\}, \delta^{E}, q_0^{E}, F^{E})$} s.t: \newline
\begin{itemize}
  \item $Q^{E} = Q^{D} \cup \{q_0^{E}\}$
  \item $F^{E} = F^{D}$
  \item \(
    \hspace{0mm}
    \delta^{E}(q, \sigma) = 
    \begin{cases}
      \delta^{D}(q, \sigma) & q \in Q^{D} \\
        q_0^{D} & q = q_0^{E}, \sigma = a \\ 
        q_0^{E} & q = q_0^{E}, \sigma = b
    \end{cases}
  \)
\end{itemize}

Define: 
\begin{itemize}
  \item L(A)
\end{itemize}
\end{p}
\begin{s} \newline
\emph{It would be non but rational to divide this construction into three divisions, each corresponding to a different set of circumstances recognised by the trasitions function.} \newline
\\
\emph{One of those aforementioned circumstances is $q = q_0^{E}, \sigma = b$, the study of such case lead me to determine that for the character input of b, under the assumption that the current state is $q_0^{E}$, the state would lead back to itself, meaning that that an instance of $\{b\}^{*}$ at the beginning of the input would not affect the output of the automata. And hence $\{b\}^{*}$ should be imbued to the language L(A).} \newline
\\
\emph{Another set of circumstances is $q = q_0^{E}, \sigma = a$, which implies the current state to be the one added to $Q^{D}$ in order to craft $Q^{E}$, and that the input chracter is 'a'. Such circumstances appear to be digested by the automata to return $q_0^{D}$, the first state of the previous automata D. Accordingly, it would only be after the appearence of an 'a' character in the input that the state would be changed. And hence, $\{a\}$ must be added to the language L(A).} \newline
\\
\emph{The last of such circumstances addressed in $\delta^{E}$ appears to be $q \in Q^{D}$. For such case, the function would make the transition from the current state to the one returned by $\delta^{D}$, accordingly, L(D) must be concatenated at the end of L(E)} \newline 
\\
\emph{Hence - I may declare that $L(A) = \{b\}^{*} \cdot \{a\} \cdot L(D)$.}
\end{s}
\end{document}
