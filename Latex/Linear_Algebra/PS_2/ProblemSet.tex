% ***********************************************************
% ******************* PHYSICS HEADER ************************
% ***********************************************************
% Version 2
\documentclass[12pt]{article}
\usepackage{tikz, calc} % draw automatas
\usetikzlibrary{automata,positioning}
% ***********************************************************
\usepackage{amsmath} % AMS Math Package
\usepackage{amsthm} % Theorem Formatting
\usepackage{amssymb}    % Math symbols such as \mathbb
\usepackage{graphicx} % Allows for eps images
\usepackage[dvips,letterpaper,margin=1in,bottom=0.7in]{geometry}
\usepackage{tensor}
\usepackage{mathtools}
\usepackage{nicematrix}
 % Sets margins and page size
\renewcommand{\labelenumi}{(\alph{enumi})} % Use letters for enumerate
% \DeclareMathOperator{\Sample}{Sample}
\let\vaccent=\v % rename builtin command \v{} to \vaccent{}
\usepackage{enumerate}
\renewcommand{\v}[1]{\ensuremath{\mathbf{#1}}} % for vectors
\newcommand{\gv}[1]{\ensuremath{\mbox{\boldmath$ #1 $}}} 
% for vectors of Greek letters
\newcommand{\uv}[1]{\ensuremath{\mathbf{\hat{#1}}}} % for unit vector
\newcommand{\abs}[1]{\left| #1 \right|} % for absolute value
\newcommand{\avg}[1]{\left< #1 \right>} % for average
\let\underdot=\d % rename builtin command \d{} to \underdot{}
\renewcommand{\d}[2]{\frac{d #1}{d #2}} % for derivatives
\newcommand{\dd}[2]{\frac{d^2 #1}{d #2^2}} % for double derivatives
\newcommand{\pd}[2]{\frac{\partial #1}{\partial #2}} 
% for partial derivatives
\newcommand{\pdd}[2]{\frac{\partial^2 #1}{\partial #2^2}} 
% for double partial derivatives
\newcommand{\pdc}[3]{\left( \frac{\partial #1}{\partial #2}
 \right)_{#3}} % for thermodynamic partial derivatives
\newcommand{\ket}[1]{\left| #1 \right>} % for Dirac bras
\newcommand{\bra}[1]{\left< #1 \right|} % for Dirac kets
\newcommand{\braket}[2]{\left< #1 \vphantom{#2} \right|
 \left. #2 \vphantom{#1} \right>} % for Dirac brackets
\newcommand{\matrixel}[3]{\left< #1 \vphantom{#2#3} \right|
 #2 \left| #3 \vphantom{#1#2} \right>} % for Dirac matrix elements
\newcommand{\grad}[1]{\gv{\nabla} #1} % for gradient
\let\divsymb=\div % rename builtin command \div to \divsymb
\renewcommand{\div}[1]{\gv{\nabla} \cdot \v{#1}} % for divergence
\newcommand{\curl}[1]{\gv{\nabla} \times \v{#1}} % for curl
\newcommand{\bigslant}[2]{{\raisebox{.2em}{$#1$}\left/\raisebox{-.2em}{$#2$}\right.}} % Quotient
\let\baraccent=\= % rename builtin command \= to \baraccent
\renewcommand{\=}[1]{\stackrel{#1}{=}} % for putting numbers above =
\providecommand{\wave}[1]{\v{\tilde{#1}}}
\providecommand{\fr}{\frac}
\providecommand{\RR}{\mathbb{R}}
\providecommand{\NN}{\mathbb{N}}
\providecommand{\seq}{\subseteq}
\providecommand{\e}{\varepsilon}

\newtheorem{prop}{Proposition}
\newtheorem{thm}{Theorem}[section]
\newtheorem{axiom}{Axiom}[section]
\newtheorem{p}{Problem}[section]
\usepackage{cancel}
\newtheorem*{lem}{Lemma}
\theoremstyle{definition}
\newtheorem*{dfn}{Definition}
 \newenvironment{s}{%\small%
        \begin{trivlist} \item \textbf{Solution}. }{%
            \hspace*{\fill} $\blacksquare$\end{trivlist}}%
% ***********************************************************
% ********************** END HEADER *************************
% ***********************************************************

\begin{document}

{\noindent\Huge\bf  \\[0.5\baselineskip] {\fontfamily{cmr}\selectfont  Problem Set II}         }\\[2\baselineskip] % Title
{ {\bf \fontfamily{cmr}\selectfont Linear Algebra}\\ {\textit{\fontfamily{cmr}\selectfont     July 9, 2023}}}~~~~~~~~~~~~~~~~~~~~~~~~~~~~~~~~~~~~~~~~~~~~~~~~~~~~~~~~~~~~~~~~~~~~~~~~~~~~~    {\large \textsc{Alon Filler}\footnote{Sorer}} % Author name
\\[1.4\baselineskip] 
\section{:}
\begin{p}
  \emph{Find the canonical row echelon form of the matrix:} \\
  \\
  \[
  \begin{pmatrix}
    7 & 7 & 4 & 28 & 21 \\
    3 & 3 & 1 & 12 & 9  \\
    6 & 5 & 2 & 20 & 18 \\
    2 & 3 & 1 & 12 & 6  \\
    4 & 6 & 4 & 24 & 12 \\
  \end{pmatrix}
  \]
\end{p}
\begin{s} \newline \\
  $\xRightarrow{\text{swap $R_1$ with $R_5$}}$
  
  \[
    \begin{pmatrix}
      4 & 6 & 4 & 24 & 12 \\
      3 & 3 & 1 & 12 & 9  \\
      6 & 5 & 2 & 20 & 18 \\
      2 & 3 & 1 & 12 & 6  \\
      7 & 7 & 4 & 28 & 21 \\
    \end{pmatrix}
  \]
  \\
  $\xRightarrow{\text{Divide the new $R_1$ by $R_{1,1}$}}$
  
  \[
    \begin{pmatrix}
      1 & 1.5 & 1 & 6 & 3 \\
      3 & 3 & 1 & 12 & 9  \\
      6 & 5 & 2 & 20 & 18 \\
      2 & 3 & 1 & 12 & 6  \\
      7 & 7 & 4 & 28 & 21 \\
    \end{pmatrix}
  \]
  \\
  $\xRightarrow{\text{From each row $R_{i > 1}$ subtract $R_i * R_1$}}$
  
  \[
    \begin{pmatrix}
      1 & 1.5 & 1 & 6 & 3 \\
      0 & -1.5 & -2   & -6 & 0  \\
      0 & -4 & -4 & -16  & 0 \\
      0 & 0 & -1 & 0  & 0  \\
      0 & -3.5 & -3 & -14  & 0 \\
    \end{pmatrix}
  \]
  \\
  $\xRightarrow{\text{swap $R_2$ with $R_3$}}$
  
  \[
    \begin{pmatrix}
      1 & 1.5 & 1 & 6 & 3 \\
      0 & -4 & -4 & -16  & 0 \\
      0 & -1.5 & -2   & -6 & 0  \\
      0 & 0 & -1 & 0  & 0  \\
      0 & -3.5 & -3 & -14  & 0 \\
    \end{pmatrix}
  \]
  \\
  $\xRightarrow{\text{Divide the new $R_2$ by $R_{2,2}$}}$
  
  \[
    \begin{pmatrix}
      1 & 1.5 & 1 & 6 & 3 \\
      0 & 1 & 1 & 4  & 0 \\
      0 & -1.5 & -2   & -6 & 0  \\
      0 & 0 & -1 & 0  & 0  \\
      0 & -3.5 & -3 & -14  & 0 \\
    \end{pmatrix}
  \]
  \\
  $\xRightarrow{\text{From each row $R_{i > 2}$ subtract $R_i * R_2$}}$
  
  \[
    \begin{pmatrix}
      1 & 1.5 & 1 & 6 & 3 \\
      0 & 1 & 1 & 4  & 0 \\
      0 & 0 & -0.5 & 0 & 0  \\
      0 & 0 & -1 & 0  & 0  \\
      0 & 0 & 0.5 & 0  & 0 \\
    \end{pmatrix}
  \]
  \\
  $\xRightarrow{\text{Multiply $R_3$ by -2}}$
  
  \[
    \begin{pmatrix}
      1 & 1.5 & 1 & 6 & 3 \\
      0 & 1 & 1 & 4  & 0 \\
      0 & 0 & 1 & 0 & 0  \\
      0 & 0 & -1 & 0  & 0  \\
      0 & 0 & 0.5 & 0  & 0 \\
    \end{pmatrix}
  \]
  \\
  $\xRightarrow{\text{From each row $R_{i > 3}$ subtract $R_i * R_3$}}$
  
  \[
    \begin{pmatrix}
      1 & 1.5 & 1 & 6 & 3 \\
      0 & 1 & 1 & 4  & 0 \\
      0 & 0 & 1 & 0 & 0  \\
      0 & 0 & 0 & 0  & 0  \\
      0 & 0 & 0 & 0  & 0 \\
    \end{pmatrix}
  \]
  \\
  
  $\xRightarrow{\text{From each row $R_{i < 2}$ subtract $R_{i,2} * R_2$}}$
  
  \[
    \begin{pmatrix}
      1 & 0 & -0.5 & 0 & 3 \\
      0 & 1 & 1 & 4  & 0 \\
      0 & 0 & 1 & 0 & 0  \\
      0 & 0 & 0 & 0  & 0  \\
      0 & 0 & 0 & 0  & 0 \\
    \end{pmatrix}
  \]
  \\
  $\xRightarrow{\text{From each row $R_{i < 3}$ subtract $R_{i,3} * R_3$}}$
  
  \[
    \begin{pmatrix}
      1 & 0 & 0 & 0 & 3 \\
      0 & 1 & 0 & 4  & 0 \\
      0 & 0 & 1 & 0 & 0  \\
      0 & 0 & 0 & 0  & 0  \\
      0 & 0 & 0 & 0  & 0 \\
    \end{pmatrix}
  \]
  \\
\end{s}
\begin{p}
  \emph{Solve the System of Equations above $\mathbb{R}$:} \\
  \\
  \[
    \begin{array}{lcl} 
    2x +7y + 13z & = & 33 \\ 
    2x + 4y + 7z & = & 15 \\ 
    1x + 2y + 4z & = & 7 \\
    \end{array}
  \]
\end{p}
\begin{s} \newline \\
  $\xRightarrow{\text{Display as Matrix}}$
  \\
     
  \[
    \left(\begin{NiceArray}{ccc|c}
      2 & 7 & 13 & 33 \\
      2 & 4 & 7  & 15 \\
      1 & 2 & 4  & 7 \\
    \end{NiceArray} \right)
  \]
  \\
  $\xRightarrow{\text{$R_2 = R_2 - R_1$}}$
  \\
     
  \[
    \left(\begin{NiceArray}{ccc|c}
      2 & 7  & 13 & 33  \\
      0 & -3 & -6 & -18 \\
      1 & 2  & 4  & 7   \\
    \end{NiceArray} \right)
  \]
  \\
  $\xRightarrow{\text{$R_3 = 2 * R_3 - R_1$}}$
  \\
     
  \[
    \left(\begin{NiceArray}{ccc|c}
      2 & 7  & 13 & 33  \\
      0 & -3 & -6 & -18 \\
      0 & -3 & -5 & -19 \\
    \end{NiceArray} \right)
  \]
  \\
  $\xRightarrow{\text{$R_3 = R_3 - R_2$}}$
  \\
     
  \[
    \left(\begin{NiceArray}{ccc|c}
      2 & 7  & 13 & 33  \\
      0 & -3 & -6 & -18 \\
      0 & 0  & 1  & -1  \\
    \end{NiceArray} \right)
  \]
  \\

  $\xRightarrow{\text{$R_1 = 3 * R_1 + 7 * R_2$}}$
  \\
     
  \[
    \left(\begin{NiceArray}{ccc|c}
      6 & 0  & -3 & -27  \\
      0 & -3 & -6 & -18 \\
      0 & 0  & 1  & -1  \\
    \end{NiceArray} \right)
  \]
  \\

  $\xRightarrow{\text{$R_1 = R_1 + 3 * R_3$}}$
  \\
     
  \[
    \left(\begin{NiceArray}{ccc|c}
      6 & 0  & 0  & -30 \\
      0 & -3 & -6 & -18 \\
      0 & 0  & 1  & -1  \\
    \end{NiceArray} \right)
  \]
  \\

  $\xRightarrow{\text{$R_2 = R_2 + 6 * R_3$}}$
  \\
     
  \[
    \left(\begin{NiceArray}{ccc|c}
      6 & 0  & 0  & -30 \\
      0 & -3 & 0 & -24 \\
      0 & 0  & 1  & -1  \\
    \end{NiceArray} \right)
  \]
  \\
  $\xRightarrow{\text{$R_1 = R_1 / 6, R_2 = R_2 / (-3)$}}$
  \\
     
  \[
    \left(\begin{NiceArray}{ccc|c}
      1 & 0  & 0  & -5 \\
      0 & 1 & 0 & 8 \\
      0 & 0  & 1  & -1  \\
    \end{NiceArray} \right)
  \]
  \\
  $\xRightarrow{\text{Display as System of Equations}}$
  \\
  \[
    \begin{array}{lcl} 
    2x +7y + 13z & = & 33 \\ 
    -3y + -6z & = & -18 \\ 
    1z & = & -1 \\
    \end{array}
  \]
  \\
  $\xRightarrow{\text{z = -1}}$
  \\
  \[
    \begin{array}{lcl} 
    2x +7y + 13z & = & 33 \\ 
    y & = & 8 \\ 
    z & = & -1 \\
    \end{array}
  \]
  \\
  $\xRightarrow{\text{y = 8, z = -1}}$
  \\
  \[
    \begin{array}{lcl} 
    x & = & -5 \\ 
    y & = & 8 \\ 
    z & = & -1 \\
    \end{array}
  \]
\end{s}
\begin{p}
  \emph{Given the following System of Equations, determine for which $a$ values the System has one solution, no solutions or an infinite amount of solutions:} \\

  \[
    \begin{array}{lcl} 
    x + y + az & = & 1 \\
    x + ay + z & = & 1 \\
    ax + y + z & = & 1  
    \end{array}
  \]
\end{p}
\begin{s} \newline \\
  $\xRightarrow{\text{Display as Matrix}}$
  \\
     
  \[
    \left(\begin{NiceArray}{ccc|c}
      1 & 1 & a & 1 \\
      1 & a & 1  & 1 \\
      a & 1 & 1  & 1 \\
    \end{NiceArray} \right)
  \]
  \\
  
  $\xRightarrow{\text{$R_2$ = $R_2$ - $R_1$}}$
  \\
     
  \[
    \left(\begin{NiceArray}{ccc|c}
      1 & 1 & a & 1 \\
      0 & a - 1 & 1 - a & 0 \\
      a & 1 & 1  & 1 \\
    \end{NiceArray} \right)
  \]
  \\

  $\xRightarrow{\text{$R_3$ = $R_3$ - a * $R_1$}}$
  \\
     
  \[
    \left(\begin{NiceArray}{ccc|c}
      1 & 1 & a & 1 \\
      0 & a - 1 & 1 - a & 0 \\
      0 & 1 - a & 1 - a^2 & 1 - a \\
    \end{NiceArray} \right)
  \]
  \\
  $\xRightarrow{\text{$R_3$ = $R_3$ + $R_2$}}$
  \\

  \[
    \left(\begin{NiceArray}{ccc|c}
      1 & 1     & a             & 1 \\
      0 & a - 1 & 1 - a         & 0 \\
      0 & 0     & -(a-1)(a+2) & 1 - a \\
    \end{NiceArray} \right)
  \]
  \\
  \\
  $\xRightarrow{\text{Testing Edge Cases:}}$ \\
  \\
  \begin{equation}
    \begin{cases}
      a = 1 \Rightarrow - (1-1)(1+2) = 1 - 1 \Rightarrow 0 = 0 \Rightarrow x,y,z \in \mathbb{R} \\
      a = 2 \Rightarrow - (-2-1)(-2+2) = 1 - (-2) \Rightarrow 0 = 3 \Rightarrow x,y,z \notin \mathbb{R} \\
    
    \end{cases}\
  \end{equation}
  \\
  \emph{Now, assuming a $\neq$ 1,-2 let us find general solutions} \\
  $\xRightarrow{\text{$R_1$ = $R_1$ - $ \frac{1}{a - 1} R_2$}}$
  \\
     
  \[
    \left(\begin{NiceArray}{ccc|c}
      1 & 0     & a + 1         & 1 \\
      0 & a - 1 & 1 - a         & 0 \\
      0 & 0     & -a^{2} - a + 2 & 1 - a \\
    \end{NiceArray} \right)
  \]
  \\
  $\xRightarrow{\text{$R_1$ = $R_1$ - $ \frac{a + 1}{-a^2 - a + 2} R_3$}}$
  \\
     
  \[
    \left(\begin{NiceArray}{ccc|c}
      1 & 0     & 0                  & 1 - \frac{1-a^2}{-a^2 - a + 2} \\
      0 & a - 1 & 1 - a              & 0     \\
      0 & 0     & -a^{2} - a + 2     & 1 - a \\
    \end{NiceArray} \right)
  \]
  \\
  $\xRightarrow{\text{$R_2$ = $R_2$ - $ \frac{1 - a}{-a^2 - a + 2} R_3$}}$
  \\
     
  \[
    \left(\begin{NiceArray}{ccc|c}
      1 & 0     & 0             & \frac{1}{a + 2} \\
      0 & a - 1 & 0             & - \frac{(1 - a)^2}{-a^2 - a + 2}     \\
      0 & 0     & -a^2 - a + 2  & 1 - a \\
    \end{NiceArray} \right)
  \]
  \\
  $\xRightarrow{\text{Display as System of Equations}}$
  \\
  \[
    \begin{array}{lcl} 
    x & = & \frac{1}{a + 2} \\
    (a - 1)y & = & - \frac{(1 - a)^2}{-a^2 - a + 2} \\
    (-a^2 - a + 2)z & = & 1 - a  
    \end{array}
  \]
  $\xRightarrow{\text{(2): /:(a - 1)}}$
  \\
  \[
    \begin{array}{lcl} 
    x & = & \frac{1}{a + 2} \\
    y & = & - \frac{(1 - a)}{-a^2 - a + 2} \\
    (-a^2 - a + 2)z & = & 1 - a  
    \end{array}
  \]
  $\xRightarrow{\text{(3): /:(-$a^2$ - a + 2), (2): finding roots}}$
  \\
  \[
    \begin{array}{lcl} 
    x & = & \frac{1}{a+2} \\
    y & = & \frac{1}{a+2} \\
    z & = & \frac{1}{a+2}  
    \end{array}
  \]

  $\xRightarrow{\text{z = -y}}$
  \\
  \[
    \begin{array}{lcl} 
    x & = & \frac{1}{a + 2} \\
    y & = & - \frac{1}{a+2} \\
    z & = & -y  
    \end{array}
  \]
  \\
  \emph{And hence the solution is:} \\
  \[
    \begin{pmatrix}
      \frac{1}{a+2} \\
      \frac{1}{a+2} \\
      \frac{1}{a+2}
    \end{pmatrix}
  \]
\end{s}
\end{document}
