% ***********************************************************
% ******************* PHYSICS HEADER ************************
% ***********************************************************
% Version 2
\documentclass[12pt]{article}
\usepackage{tikz, calc} % draw automatas
\usetikzlibrary{automata,positioning}
% ***********************************************************
\usepackage{amsmath} % AMS Math Package
\usepackage{amsthm} % Theorem Formatting
\usepackage{amssymb}    % Math symbols such as \mathbb
\usepackage{graphicx} % Allows for eps images
\usepackage[dvips,letterpaper,margin=1in,bottom=0.7in]{geometry}
\usepackage{tensor}
\usepackage{mathtools}
 % Sets margins and page size
\renewcommand{\labelenumi}{(\alph{enumi})} % Use letters for enumerate
% \DeclareMathOperator{\Sample}{Sample}
\let\vaccent=\v % rename builtin command \v{} to \vaccent{}
\usepackage{enumerate}
\renewcommand{\v}[1]{\ensuremath{\mathbf{#1}}} % for vectors
\newcommand{\gv}[1]{\ensuremath{\mbox{\boldmath$ #1 $}}} 
% for vectors of Greek letters
\newcommand{\uv}[1]{\ensuremath{\mathbf{\hat{#1}}}} % for unit vector
\newcommand{\abs}[1]{\left| #1 \right|} % for absolute value
\newcommand{\avg}[1]{\left< #1 \right>} % for average
\let\underdot=\d % rename builtin command \d{} to \underdot{}
\renewcommand{\d}[2]{\frac{d #1}{d #2}} % for derivatives
\newcommand{\dd}[2]{\frac{d^2 #1}{d #2^2}} % for double derivatives
\newcommand{\pd}[2]{\frac{\partial #1}{\partial #2}} 
% for partial derivatives
\newcommand{\pdd}[2]{\frac{\partial^2 #1}{\partial #2^2}} 
% for double partial derivatives
\newcommand{\pdc}[3]{\left( \frac{\partial #1}{\partial #2}
 \right)_{#3}} % for thermodynamic partial derivatives
\newcommand{\ket}[1]{\left| #1 \right>} % for Dirac bras
\newcommand{\bra}[1]{\left< #1 \right|} % for Dirac kets
\newcommand{\braket}[2]{\left< #1 \vphantom{#2} \right|
 \left. #2 \vphantom{#1} \right>} % for Dirac brackets
\newcommand{\matrixel}[3]{\left< #1 \vphantom{#2#3} \right|
 #2 \left| #3 \vphantom{#1#2} \right>} % for Dirac matrix elements
\newcommand{\grad}[1]{\gv{\nabla} #1} % for gradient
\let\divsymb=\div % rename builtin command \div to \divsymb
\renewcommand{\div}[1]{\gv{\nabla} \cdot \v{#1}} % for divergence
\newcommand{\curl}[1]{\gv{\nabla} \times \v{#1}} % for curl
\newcommand{\bigslant}[2]{{\raisebox{.2em}{$#1$}\left/\raisebox{-.2em}{$#2$}\right.}} % Quotient
\let\baraccent=\= % rename builtin command \= to \baraccent
\renewcommand{\=}[1]{\stackrel{#1}{=}} % for putting numbers above =
\providecommand{\wave}[1]{\v{\tilde{#1}}}
\providecommand{\fr}{\frac}
\providecommand{\RR}{\mathbb{R}}
\providecommand{\NN}{\mathbb{N}}
\providecommand{\seq}{\subseteq}
\providecommand{\e}{\varepsilon}

\newtheorem{prop}{Proposition}
\newtheorem{thm}{Theorem}[section]
\newtheorem{axiom}{Axiom}[section]
\newtheorem{p}{Problem}[section]
\usepackage{cancel}
\newtheorem*{lem}{Lemma}
\theoremstyle{definition}
\newtheorem*{dfn}{Definition}
 \newenvironment{s}{%\small%
        \begin{trivlist} \item \textbf{Solution}. }{%
            \hspace*{\fill} $\blacksquare$\end{trivlist}}%
% ***********************************************************
% ********************** END HEADER *************************
% ***********************************************************

\begin{document}

{\noindent\Huge\bf  \\[0.5\baselineskip] {\fontfamily{cmr}\selectfont  Problem Set II}         }\\[2\baselineskip] % Title
{ {\bf \fontfamily{cmr}\selectfont Linear Algebra}\\ {\textit{\fontfamily{cmr}\selectfont     July 9, 2023}}}~~~~~~~~~~~~~~~~~~~~~~~~~~~~~~~~~~~~~~~~~~~~~~~~~~~~~~~~~~~~~~~~~~~~~~~~~~~~~    {\large \textsc{Alon Filler}\footnote{Sorer}} % Author name
\\[1.4\baselineskip] 
\section{:}
\begin{p}
  \emph{Find the canonical row echelon form of the matrix:} \\
  \\
  \[
  \begin{pmatrix}
    7 & 7 & 4 & 28 & 21 \\
    3 & 3 & 1 & 12 & 9  \\
    6 & 5 & 2 & 20 & 18 \\
    2 & 3 & 1 & 12 & 6  \\
    4 & 6 & 4 & 24 & 12 \\
  \end{pmatrix}
  \]
\end{p}
\begin{s} \newline \\
  $\xRightarrow{\text{swap $R_1$ with $R_5$}}$
  
  \[
    \begin{pmatrix}
      4 & 6 & 4 & 24 & 12 \\
      3 & 3 & 1 & 12 & 9  \\
      6 & 5 & 2 & 20 & 18 \\
      2 & 3 & 1 & 12 & 6  \\
      7 & 7 & 4 & 28 & 21 \\
    \end{pmatrix}
  \]
  \\
  $\xRightarrow{\text{Divide the new $R_1$ by $R_{1,1}$}}$
  
  \[
    \begin{pmatrix}
      1 & 1.5 & 1 & 6 & 3 \\
      3 & 3 & 1 & 12 & 9  \\
      6 & 5 & 2 & 20 & 18 \\
      2 & 3 & 1 & 12 & 6  \\
      7 & 7 & 4 & 28 & 21 \\
    \end{pmatrix}
  \]
  \\
  $\xRightarrow{\text{From each row $R_{i > 1}$ subtract $R_i * R_1$}}$
  
  \[
    \begin{pmatrix}
      1 & 1.5 & 1 & 6 & 3 \\
      0 & -1.5 & -2   & -6 & 0  \\
      0 & -4 & -4 & -16  & 0 \\
      0 & 0 & -1 & 0  & 0  \\
      0 & -3.5 & -3 & -14  & 0 \\
    \end{pmatrix}
  \]
  \\
  $\xRightarrow{\text{swap $R_2$ with $R_3$}}$
  
  \[
    \begin{pmatrix}
      1 & 1.5 & 1 & 6 & 3 \\
      0 & -4 & -4 & -16  & 0 \\
      0 & -1.5 & -2   & -6 & 0  \\
      0 & 0 & -1 & 0  & 0  \\
      0 & -3.5 & -3 & -14  & 0 \\
    \end{pmatrix}
  \]
  \\
  $\xRightarrow{\text{Divide the new $R_2$ by $R_{2,2}$}}$
  
  \[
    \begin{pmatrix}
      1 & 1.5 & 1 & 6 & 3 \\
      0 & 1 & 1 & 4  & 0 \\
      0 & -1.5 & -2   & -6 & 0  \\
      0 & 0 & -1 & 0  & 0  \\
      0 & -3.5 & -3 & -14  & 0 \\
    \end{pmatrix}
  \]
  \\
  $\xRightarrow{\text{From each row $R_{i > 2}$ subtract $R_i * R_2$}}$
  
  \[
    \begin{pmatrix}
      1 & 1.5 & 1 & 6 & 3 \\
      0 & 1 & 1 & 4  & 0 \\
      0 & 0 & -0.5 & 0 & 0  \\
      0 & 0 & -1 & 0  & 0  \\
      0 & 0 & 0.5 & 0  & 0 \\
    \end{pmatrix}
  \]
  \\
  $\xRightarrow{\text{Multiply $R_3$ by -2}}$
  
  \[
    \begin{pmatrix}
      1 & 1.5 & 1 & 6 & 3 \\
      0 & 1 & 1 & 4  & 0 \\
      0 & 0 & 1 & 0 & 0  \\
      0 & 0 & -1 & 0  & 0  \\
      0 & 0 & 0.5 & 0  & 0 \\
    \end{pmatrix}
  \]
  \\
  $\xRightarrow{\text{From each row $R_{i > 3}$ subtract $R_i * R_3$}}$
  
  \[
    \begin{pmatrix}
      1 & 1.5 & 1 & 6 & 3 \\
      0 & 1 & 1 & 4  & 0 \\
      0 & 0 & 1 & 0 & 0  \\
      0 & 0 & 0 & 0  & 0  \\
      0 & 0 & 0 & 0  & 0 \\
    \end{pmatrix}
  \]
  \\
  
  $\xRightarrow{\text{From each row $R_{i < 2}$ subtract $R_{i,2} * R_2$}}$
  
  \[
    \begin{pmatrix}
      1 & 0 & -0.5 & 0 & 3 \\
      0 & 1 & 1 & 4  & 0 \\
      0 & 0 & 1 & 0 & 0  \\
      0 & 0 & 0 & 0  & 0  \\
      0 & 0 & 0 & 0  & 0 \\
    \end{pmatrix}
  \]
  \\
  $\xRightarrow{\text{From each row $R_{i < 3}$ subtract $R_{i,3} * R_3$}}$
  
  \[
    \begin{pmatrix}
      1 & 0 & 0 & 0 & 3 \\
      0 & 1 & 0 & 4  & 0 \\
      0 & 0 & 1 & 0 & 0  \\
      0 & 0 & 0 & 0  & 0  \\
      0 & 0 & 0 & 0  & 0 \\
    \end{pmatrix}
  \]
  \\
\end{s}
\begin{p}
  \emph{Calculate the maximum height the rocket would reach:}
\end{p}
\begin{s} \newline
  \emph{Firstly, the height reached once the engine failed must be calculated. In order to calculate the latter, one must observe the following equation: $x(t) = x_0 + v_0(t - t_0) + 0.5at^2$. Applying the equation to the scenario described, we may set $x_0 = 0, v_0 = 0, t_0 = 0, t = 5, a = 30$ and therefore: $x(5) = 0 + 0(5 - 0) + 0.5 * 30 * 5^2 = 15 * 25 = 375_{m}$ \newline Now, we may calculate the distance that the rocket travelled after the failure of the engine:\newline To do so, we must first calculate the duration of time the rocket's height resumed increasing despite the engine no longer functioning. \newline We may do so using the following system of equations: $v(t) = 0 \wedge v(t) = v_0 + a(t - t_0)$, as $v_0 = 150, a = -9.87, t_0 = 5$, therefore: $0 = 150 - 9.87(t - 5) \Longrightarrow t = 20.19$ \newline. We shall use the following equation: $x(t) = x_0 + v_0(t - t_0) + 0.5at^{2}$, we may define the parameters to suit our needs: $x_0 = 375_m, v_0 = 150_{\bigslant{m}{s^{2}}}, t_0 = 5, t = 20.19, a = -9.87 \Longrightarrow x(20.19) = 375 + 150 * (20.19 - 5) + 0.5(-9.87)(15.19)^2 = 1514.8173465$}

\end{s}
\end{document}
